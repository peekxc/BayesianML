\documentclass{tufte-handout}
\usepackage{graphicx}
\usepackage{exercise}
\usepackage{enumerate}
\usepackage{amsmath,amssymb,amsthm}
\author{Day X: Catchy Title}
\date{*CLASS DATE*}

\begin{document}


\vspace{0.1in}

% Project Report Title
% Name:
% UID: 

% Goals [why?]: what problem/challenge motivates this work? What value it brings? Goals can be broader than what you will achieve in this project.

% Objective [What?]: Specific, demonstrable or measurable outcomes/capabilities/techniques/...identify if there is any innovation/novelty

\section{Approach} Discuss steps take, technology/method chosen, data obtained/used, testing done; provide architecture schematic; identify how you obtained innovation/novelty

% Evaluation and/or Demo and/or Outcome: how did you test/evaluate? provide snapshots if appropriate to demonstrate how objectives were met.

% Conclusion

% References

% Appendix: Schedule and Milestones

% \section{Overview and Orientation}
% This is a high-level overview of the goals and plan for the day. This is only a paragraph in length, but establishes expectations for whether people should work in groups, how far they should expect to get, and (if possible) how this class session ties into other building blocks and projects.


\section{Some Content}
% \begin{marginfigure}
% Extra Resources:
% \begin{itemize}
% \item here are some links to resources
% \item there might be videos by the teaching team
% \item or by Khan Academy or something
% \item there could also be some cool written resources
% \item and extra problems for if you get stuck or if you want to go into more depth
% \end{itemize}
% \end{marginfigure}

Here there is some written explanation of content and problems to work on. To the extent possible, include
% \begin{lstlisting}
% %code examples
% %in pretty boxes
% \end{lstlisting}

\begin{enumerate}
\item Now we start the problems
\end{enumerate}


\end{document}
